\documentclass[svgnames]{article}
\usepackage[utf8]{inputenc} 
\usepackage{polski}       
\usepackage{a4wide}
\usepackage{graphicx}
\usepackage{amsmath,amssymb}
\usepackage{bbm}            % sudo apt-get install texlive-fonts-recommended texlive-fonts-extra
\usepackage{amsthm}
\usepackage{algorithmic}	% sudo apt-get install texlive-science
\usepackage{listings}             % Include the listings-package
\usepackage{framed}
\usepackage{enumerate}
\usepackage{fancyhdr}
\usepackage{amsmath, amsthm, amssymb}

\makeatletter
 \renewcommand\@seccntformat[1]{\csname  the#1\endcsname.\quad}
 
 \thispagestyle{fancy}
\begin{document}
%\tableofcontents

%%%%%%%%%%%%%%%%%%%%%%%%%%%%%%%%%%%%%%%%%%%%%%%%%%%%%%%%%%%%%%%%%%%%%%%%%%%%%%%%%%%%%%%%%%%%%%%%%%%%%%%%

\section{Lista 1}
\rhead{Tomasz Łoszko}
\subsection{}%1
\subsection{}%2
\subsection{}%3
\subsection{}%4
\subsection{}%5
\subsection{}%6 **************
\begin{framed}
Wykaż, że  $e^{\frac{1}{n}} = 1 + \frac{1}{n} + O(\frac{1}{n^2})$.
\end{framed}

\begin{equation}
e^x = \sum\limits_{i=0}^{+\inf} \frac{x^i}{i!}
\end{equation}

\begin{equation}
e^{\frac{1}{n}} = \sum\limits_{i=0}^{+\inf} \frac{ (\frac{1}{n})^i }{i!} = \sum\limits_{i=0}^{+\inf} \frac{1}{i! \cdot n^i} = 1 + \frac{1}{n} + \sum\limits_{i=2}^{+\inf} \frac{1}{i! \cdot n^i} = 1 + \frac{1}{n} + O(\frac{1}{n^2})
\end{equation}


\begin{equation*}
	\sum\limits_{i=2}^{+\inf} \frac{1}{i! \cdot n^i} \in O(\frac{1}{n^2})
\end{equation*}
\begin{proof}

\begin{equation*}
\sum\limits_{i=2}^{+\inf} \frac{1}{i! \cdot n^i} \leqslant \sum\limits_{i=2}^{+\inf} \frac{1}{2 \cdot n^i} = \frac{1}{2} \cdot \sum\limits_{i=2}^{+\inf} \frac{1}{n^i} = \frac{1}{2} \cdot \frac{ \frac{1}{n} }{ 1 - \frac{1}{n} } = \frac{1}{2 \cdot n \cdot (n-1)} \in O(\frac{1}{n^2})
\end{equation*}

\end{proof}


\subsection{}%7 **************
\begin{framed}
treść
\end{framed}

\subsection{}%8 ***************
\begin{framed}
treść
\end{framed}

\subsection{}%9

\subsection{}%10 ****************
\begin{framed}
treść
\end{framed}

\subsection{}%11
\subsection{}%12

\subsection{}%13 ****************
\begin{framed}
treść
\end{framed}

\subsection{}%14 ***************
\begin{framed}
Pokaż, że dla dowolnego $x \leq 0$:
$$ \left\lfloor \sqrt{x} \right\rfloor = \left\lfloor \sqrt{ \lfloor x \rfloor} \right\rfloor $$
\end{framed}

\begin{proof}
$ \left\lfloor \sqrt{ \lfloor x \rfloor} \right\rfloor = n $ gdzie $ n \in \mathbb{N} $ i $ n \geq 0 $. 
Z definicji podłogi: 
$$ n \leq \sqrt{ \lfloor x \rfloor} < n+1 \Leftrightarrow n^2 \leq \lfloor x \rfloor < (n+1)^2 $$

Ponieważ $n^2$ i $(n+1)^2$ są liczbami naturalnymi to w oczywisty sposób zachodzi:

$ n^2 \leq \lfloor x \rfloor \Leftrightarrow n^2 \leq x $, oraz:
$ \lfloor x \rfloor < (n+1)^2 \Leftrightarrow x <  (n+1)^2$

Zatem:
$$  n^2 \leq x <  (n+1)^2 $$
Ponieważ $x > 0$ i $n > 0$, oraz korzystając z definicji podłogi:
$$  n \leq \sqrt{x} <  n+1 \Leftrightarrow  \left\lfloor \sqrt{ \lfloor x \rfloor} \right\rfloor = n $$

Zatem:

$$ n \leq \sqrt{ \lfloor x \rfloor} < n+1 \Leftrightarrow n^2 \leq \lfloor x \rfloor < (n+1)^2 $$

\end{proof}

\subsection{}%15
%%%%%%%%%%%%%%%%%%%%%%%%%%%%%%%%%%%%%%%%%%%%%%%%%%%%%%%%%%%%%%%%%%%%%%%%%%%%%%%%%%%%%%%%%%%%%%%%%%%%%%%%


\end{document}



